\documentclass[12pt]{article}
\usepackage[a4paper, margin=1cm]{geometry}

\author{Alex W, Patryk M, Eoghan M, Michael M}
\title{Acoustic Tracing}
\setlength{\parskip}{0.75em}
\setlength{\parindent}{0pt}

\begin{document}
\maketitle
\section*{Project Proposal}

\subsection*{Introduction}
Initial idea is to trace sound waves from a sound source placed inside an environment, and to render and observe how the sound travels through the space and interacts with the environments surfaces.

Be able to place an observer inside the environment, constructed with an \texttt{.obj} file. Which later can be retrieved from using the LiDar scanner on a mobile device.

For the first iteration of the project, we are looking for the ability to simulate a room/environment along with the sound waves from the source, and how they interact with the environment and how they are interpreted by an observer. We will have a fixed source and observer along with a fixed viewpoint of the scene.

For later iterations we want to be able to take into account the reflectiveness of the materials and how they alter the intensity and direction of the sound waves as they travel through the space. We want to add interactivity to the placement of the source and the observer and potential maneuverability throughout the scene.

\subsection*{Desired Features}
    \begin{itemize}
        \item ML Material Classification
        \item Observer Sound Intensity
        \item Optimal Sound Source Location
        \item 3D Scene Renderer
        \item Acoustic Rasterizer
        \item React Frontend with user-uploaded \texttt{.obj} files
    \end{itemize} 

\subsection*{Implementation Stack and Technologies}
    \begin{description}
        \item [C] Performant Raytracing
        \item [Python] File handling, \texttt{.obj} parsing and simulation handling
        \item [React] Modern frontend and user interactivity
    \end{description}
    
\end{document}